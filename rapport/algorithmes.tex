% !TEX root = ./main.tex
%%%%%%%%%%%%%%%%%%%%%%%%%%%%%%%%%%%%%%%%%%%%%%%%%%%%%%%%%%%%%%%%%%%%%%%%%%%%%%%%%%%%%%%%%%
% Vous pourriez être amenés, dans ce projet, à développer des algorithmes poussés,       %
% notamment pour le déroulement du jeu. Décrivez l'idée de ces algorithmes dans votre    %
% rapport et comment vous les avez implémentés (structures de données particulières,     %
% fonctionnement général, ...)                                                           %
%%%%%%%%%%%%%%%%%%%%%%%%%%%%%%%%%%%%%%%%%%%%%%%%%%%%%%%%%%%%%%%%%%%%%%%%%%%%%%%%%%%%%%%%%%
\section{Algorithmes}\label{algorithmes}
%%%%%%%%%%%%%%%%%%%%%%

\begin{itemize}
    \item Rotation d'image: 
        Dans notre projet une image est représentée par des matrices, notamment pour la représentation de des pieces.
    \item Tri d'une liste doublement chainée: 
        Pour stoquer les différents scores dans notre jeu, nous avons utilisé une liste doublement chainée, et il
        a fallu implémenter un algorithme de trie pour cette liste.
    \item Detection des collisions: Pour la detection des collisions entre les pieces.
\end{itemize}