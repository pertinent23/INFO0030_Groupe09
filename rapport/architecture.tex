% !TEX root = ./main.tex
%%%%%%%%%%%%%%%%%%%%%%%%%%%%%%%%%%%%%%%%%%%%%%%%%%%%%%%%%%%%%%%%%%%%%%%%%%%%%%%%%%%%%%%%%%
% Dans cette section, décrivez l'architecture générale de votre code.  Quels sont les    %
% grands concepts de votre code et comment ils interagissent entre eux.                  %
%%%%%%%%%%%%%%%%%%%%%%%%%%%%%%%%%%%%%%%%%%%%%%%%%%%%%%%%%%%%%%%%%%%%%%%%%%%%%%%%%%%%%%%%%%
\section{Architecture}\label{architecture}
%%%%%%%%%%%%%%%%%%%%%%%

\subsection[short]{Architecture Générale}

L'architecture générale du code, est l'architecture MVC (modèle, vue, controleur).

\subsection[short]{Grand Concepts}

Notre code est divisé en trois grandes parties:
\begin{itemize}
    \item Le composant \textbf{app}
    \item Le composant \textbf{pieces}
    \item Le composant \textbf{modal}
\end{itemize}

\subsubsection[short]{Le Composant App}

C'est le composant principal de l'application, il représente toute l'interface graphique 
primaire (les bouttons, les lables, les menus, etc.).