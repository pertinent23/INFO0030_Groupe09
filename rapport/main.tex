%%%%%%%%%%%%%%%%%%%%%%%%%%%%%%%%%%%%%%%%%%%%%%%%%%%%%%%%%%%%%%%%%%%%%%%%%%%%%%%%%%%%%%%%%%
% Ceci est le fichier principal du template template à utiliser pour les rapports du     %
% projet 4 d'INFO0030.                                                                   %
%                                                                                        %
% Vous devez décommenter et compléter les commandes introduites plus bas (intitule, ...) %
% avant de pouvoir compiler le fichier LaTeX.  Pensez à configurer votre Makefile en     %
% conséquence.                                                                           %
%                                                                                        %
% Le contenu et la structure du rapport sont imposés.  Vous devez compléter les          %
% différents fichiers .tex inclus dans ce fichier avec votre production.                 %
%%%%%%%%%%%%%%%%%%%%%%%%%%%%%%%%%%%%%%%%%%%%%%%%%%%%%%%%%%%%%%%%%%%%%%%%%%%%%%%%%%%%%%%%%%

% !TEX root = ./main.tex
% !TEX engine = latexmk -pdf
% !TEX buildOnSave = true
\documentclass[a4paper, 11pt, oneside]{article}

\usepackage[utf8]{inputenc}
\usepackage[T1]{fontenc}
\usepackage[french]{babel}
\usepackage{array}
\usepackage{shortvrb}
\usepackage{listings}
\usepackage[fleqn]{amsmath}
\usepackage{amsfonts}
\usepackage{fullpage}
\usepackage{enumerate}
\usepackage{graphicx}             % import, scale, and rotate graphics
\usepackage{subfigure}            % group figures
\usepackage{alltt}
\usepackage{url}
\usepackage{indentfirst}
\usepackage{eurosym}
\usepackage{listings}
\usepackage{color}
\usepackage[table,xcdraw,dvipsnames]{xcolor}

% Change le nom par défaut des listing
\renewcommand{\lstlistingname}{Extrait de Code}


\definecolor{mygray}{rgb}{0.5,0.5,0.5}
\newcommand{\coms}[1]{\textcolor{MidnightBlue}{#1}}

\lstset{
    language=C, % Utilisation du langage C
    commentstyle={\color{MidnightBlue}}, % Couleur des commentaires
    frame=single, % Entoure le code d'un joli cadre
    rulecolor=\color{black}, % Couleur de la ligne qui forme le cadre
    stringstyle=\color{RawSienna}, % Couleur des chaines de caractères
    numbers=left, % Ajoute une numérotation des lignes à gauche
    numbersep=5pt, % Distance entre les numérots de lignes et le code
    numberstyle=\tiny\color{mygray}, % Couleur des numéros de lignes
    basicstyle=\tt\footnotesize,
    tabsize=3, % Largeur des tabulations par défaut
    keywordstyle=\tt\bf\footnotesize\color{Sepia}, % Style des mots-clés
    extendedchars=true,
    captionpos=b, % sets the caption-position to bottom
    texcl=true, % Commentaires sur une ligne interprétés en Latex
    showstringspaces=false, % Ne montre pas les espace dans les chaines de caractères
    escapeinside={(>}{<)}, % Permet de mettre du latex entre des <( et )>.
    inputencoding=utf8,
    literate=
  {á}{{\'a}}1 {é}{{\'e}}1 {í}{{\'i}}1 {ó}{{\'o}}1 {ú}{{\'u}}1
  {Á}{{\'A}}1 {É}{{\'E}}1 {Í}{{\'I}}1 {Ó}{{\'O}}1 {Ú}{{\'U}}1
  {à}{{\`a}}1 {è}{{\`e}}1 {ì}{{\`i}}1 {ò}{{\`o}}1 {ù}{{\`u}}1
  {À}{{\`A}}1 {È}{{\`E}}1 {Ì}{{\`I}}1 {Ò}{{\`O}}1 {Ù}{{\`U}}1
  {ä}{{\"a}}1 {ë}{{\"e}}1 {ï}{{\"i}}1 {ö}{{\"o}}1 {ü}{{\"u}}1
  {Ä}{{\"A}}1 {Ë}{{\"E}}1 {Ï}{{\"I}}1 {Ö}{{\"O}}1 {Ü}{{\"U}}1
  {â}{{\^a}}1 {ê}{{\^e}}1 {î}{{\^i}}1 {ô}{{\^o}}1 {û}{{\^u}}1
  {Â}{{\^A}}1 {Ê}{{\^E}}1 {Î}{{\^I}}1 {Ô}{{\^O}}1 {Û}{{\^U}}1
  {œ}{{\oe}}1 {Œ}{{\OE}}1 {æ}{{\ae}}1 {Æ}{{\AE}}1 {ß}{{\ss}}1
  {ű}{{\H{u}}}1 {Ű}{{\H{U}}}1 {ő}{{\H{o}}}1 {Ő}{{\H{O}}}1
  {ç}{{\c c}}1 {Ç}{{\c C}}1 {ø}{{\o}}1 {å}{{\r a}}1 {Å}{{\r A}}1
  {€}{{\euro}}1 {£}{{\pounds}}1 {«}{{\guillemotleft}}1
  {»}{{\guillemotright}}1 {ñ}{{\~n}}1 {Ñ}{{\~N}}1 {¿}{{?`}}1
}
\newcommand{\tablemat}{~}

%%%%%%%%%%%%%%%%% TITRE %%%%%%%%%%%%%%%%
% Complétez et décommentez les définitions de macros suivantes :
% \newcommand{\intitule}{Le titre du travail}
% \newcommand{\GrNbr}{1742}
% \newcommand{\PrenomUN}{Galileo}
% \newcommand{\NomUN}{Galilei}
% \newcommand{\PrenomDEUX}{Octave}
% \newcommand{\NomDEUX}{Urbain}

\renewcommand{\tablemat}{\tableofcontents}

%%%%%%%% ZONE PROTÉGÉE : MODIFIEZ UNE DES DIX PROCHAINES %%%%%%%%
%%%%%%%%            LIGNES POUR PERDRE 2 PTS.            %%%%%%%%
\title{INFO0030: \intitule}
\author{Groupe \GrNbr : \PrenomUN~\textsc{\NomUN}, \PrenomDEUX~\textsc{\NomDEUX}}
\date{}
\begin{document}

\maketitle
\newpage
\tablemat
\newpage

%%%%%%%%%%%%%%%% RAPPORT %%%%%%%%%%%%%%%

% Inclusion des différentes sections

% !TEX root = ./main.tex
%%%%%%%%%%%%%%%%%%%%%%%%%%%%%%%%%%%%%%%%%%%%%%%%%%%%%%%%%%%%%%%%%%%%%%%%%%%%%%%%%%%%%%%%%%
% Rédigez ici l'introduction de votre rapport.                                           %
%%%%%%%%%%%%%%%%%%%%%%%%%%%%%%%%%%%%%%%%%%%%%%%%%%%%%%%%%%%%%%%%%%%%%%%%%%%%%%%%%%%%%%%%%%
\section{Introduction}\label{introduction}
%%%%%%%%%%%%%%%%%%%%%%%

Le Tetris est un jeu 2D (en deux dimensiosn) qui consiste à déplacer des formes, à les 
emboiter pour former des lignes parfaites qui permetrons à l'utilisateur de gagner des points.

Dans ce projet nous proposerons une version simplifié du Tetris, les joueurs pourrons déplacer
les pieces, les agencers et gagner des points en fonction du nombre de lignes qu'ils pourront former.

% !TEX root = ./main.tex
%%%%%%%%%%%%%%%%%%%%%%%%%%%%%%%%%%%%%%%%%%%%%%%%%%%%%%%%%%%%%%%%%%%%%%%%%%%%%%%%%%%%%%%%%%
% Dans cette section, décrivez l'architecture générale de votre code.  Quels sont les    %
% grands concepts de votre code et comment ils interagissent entre eux.                  %
%%%%%%%%%%%%%%%%%%%%%%%%%%%%%%%%%%%%%%%%%%%%%%%%%%%%%%%%%%%%%%%%%%%%%%%%%%%%%%%%%%%%%%%%%%
\section{Architecture}\label{architecture}
%%%%%%%%%%%%%%%%%%%%%%%

\subsection[short]{Architecture Générale}

L'architecture générale du code, est l'architecture MVC (modèle, vue, controleur).

\subsection[short]{Grand Concepts}

Notre code est divisé en trois grandes parties:
\begin{itemize}
    \item Le composant \textbf{app}
    \item Le composant \textbf{pieces}
    \item Le composant \textbf{modal}
\end{itemize}

\subsubsection[short]{Le Composant App}

C'est le composant principal de l'application, il représente toute l'interface graphique 
primaire (les bouttons, les lables, les menus, etc.).

% !TEX root = ./main.tex
%%%%%%%%%%%%%%%%%%%%%%%%%%%%%%%%%%%%%%%%%%%%%%%%%%%%%%%%%%%%%%%%%%%%%%%%%%%%%%%%%%%%%%%%%%
% Dans cette section, décrivez les structures de données mises en place.                 %
% Pensez à discuter la pertinence et/ou le coût de ces structures.                       %
%%%%%%%%%%%%%%%%%%%%%%%%%%%%%%%%%%%%%%%%%%%%%%%%%%%%%%%%%%%%%%%%%%%%%%%%%%%%%%%%%%%%%%%%%%
\section{Structures de Données}\label{structures}
%%%%%%%%%%%%%%%%%%%%%%%%%%%%%%%%

Pour chaque composants principaux, nous avons mis en place 03 tructures qui répresentent pour chacun,
leur modèle, vue et controleur.

Ainsi les tructures principales sont:
\begin{itemize}
    \item App:
        \item AppModel_t
        \item AppVue_t
        \item AppControleur_t
    \item Modal
        \iitem ModalModel_t
        \iitem ModalVue_t
        \iitem ModalControleur_t
    \item Piece
\end{itemize}

% !TEX root = ./main.tex
%%%%%%%%%%%%%%%%%%%%%%%%%%%%%%%%%%%%%%%%%%%%%%%%%%%%%%%%%%%%%%%%%%%%%%%%%%%%%%%%%%%%%%%%%%
% Vous pourriez être amenés, dans ce projet, à développer des algorithmes poussés,       %
% notamment pour le déroulement du jeu. Décrivez l'idée de ces algorithmes dans votre    %
% rapport et comment vous les avez implémentés (structures de données particulières,     %
% fonctionnement général, ...)                                                           %
%%%%%%%%%%%%%%%%%%%%%%%%%%%%%%%%%%%%%%%%%%%%%%%%%%%%%%%%%%%%%%%%%%%%%%%%%%%%%%%%%%%%%%%%%%
\section{Algorithmes}\label{algorithmes}
%%%%%%%%%%%%%%%%%%%%%%


% !TEX root = ./main.tex
%%%%%%%%%%%%%%%%%%%%%%%%%%%%%%%%%%%%%%%%%%%%%%%%%%%%%%%%%%%%%%%%%%%%%%%%%%%%%%%%%%%%%%%%%%
% Fournissez des captures d'écran de votre interface graphique et commentez-les.         %
% Expliquez comment vous avez organisé votre jeu (table, box, ...).                      %
%%%%%%%%%%%%%%%%%%%%%%%%%%%%%%%%%%%%%%%%%%%%%%%%%%%%%%%%%%%%%%%%%%%%%%%%%%%%%%%%%%%%%%%%%%
\section{Interface Graphique}\label{ihm}
%%%%%%%%%%%%%%%%%%%%%%%%%%%%%


% !TEX root = ./main.tex
%%%%%%%%%%%%%%%%%%%%%%%%%%%%%%%%%%%%%%%%%%%%%%%%%%%%%%%%%%%%%%%%%%%%%%%%%%%%%%%%%%%%%%%%%%
% Expliquez comment vous avez géré le GIT et ce que cela vous a apporté tout au long du  %
% projet.                                                                                %
%%%%%%%%%%%%%%%%%%%%%%%%%%%%%%%%%%%%%%%%%%%%%%%%%%%%%%%%%%%%%%%%%%%%%%%%%%%%%%%%%%%%%%%%%%
\section{Gestion du Code}\label{scm}
%%%%%%%%%%%%%%%%%%%%%%%%%%%%%


% !TEX root = ./main.tex
%%%%%%%%%%%%%%%%%%%%%%%%%%%%%%%%%%%%%%%%%%%%%%%%%%%%%%%%%%%%%%%%%%%%%%%%%%%%%%%%%%%%%%%%%%
% Rédigez ici la conclusion de votre rapport.  En particulier, vous devez:               %
% - Expliquer comment votre binôme a fonctionné et comment vous avez géré votre          %
%   coopération.                                                                         %
% - Décrire les améliorations possibles à votre application (par exemple, si vous aviez  %
%   disposé d'un mois supplémentaire).                                                   %
% - Décrire ce que ce projet vous a apporté et ce que vous en avez appris.               %
%%%%%%%%%%%%%%%%%%%%%%%%%%%%%%%%%%%%%%%%%%%%%%%%%%%%%%%%%%%%%%%%%%%%%%%%%%%%%%%%%%%%%%%%%%
\section{Conclusion}\label{conclusion}
%%%%%%%%%%%%%%%%%%%%%


%%%%%%%%%%%%%%%%%%%% FIN DE LA ZONE PROTÉGÉE %%%%%%%%%%%%%%%%%%%%

\end{document}
